\chapter{Literature Review}
\label{chapterlabel2}

In this literature review, we will introduce the OpenStreetMap project, situating it within the broader phenomena of volunteered geographic information, neogeography, and Web 2.0. We will then address key issues relating to data quality in OSM and review the large volume of past work that has addressed this topic. We highlight the trend of work that has moved from extrinsic to intrinsic quality assessments and identify a need to address temporal accuracy of data in greater depth. We next focus more closely on the issue of temporal accuracy in OSM and discuss the dynamics of editing. We then focus on the case of humanitarian mapping and discuss the applications of OSM in humanitarian contexts and the unique modes of data production in this domain. We conclude by situating the work of this thesis in the research gap that exists at the intersection of OSM temporal data quality/data maintenance and humanitarian mapping efforts. 

\section{Introduction to OSM}


\section{Data Quality and OSM}

Geospatial data quality is made up of a number of attributes. Also, very important to adapt quality assessments for the task at hand. What is needed for routing, for example, may not necessarily be relevant for general spatial awareness. \textcite{fox_notion_1994}, for example, cite accuracy, completeness, currentness, and consistency as the primary dimensions of data quality. \textcite{antoniou_measures_2015} review past assessments of VGI data quality, broken down into the following categories: completeness, logical consistency, positional accuracy, temporal accuracy, thematic accuracy, and usability. In a comprehensive review, \textcite{van_oort_spatial_2006} identifies 11 elements of spatial data quality. 

Data quality is the most frequently researched topic with respect to OSM. Questions of data quality are also particularly relevant and challenging to address in this context because of the highly diverse nature of contributions and contributors. Quality across the whole database is highly heterogeneous \parencite{grochenig_estimating_2014, haklay_how_2010, neis_analyzing_2012}. The data itself is also highly heterogeneous, from many different contributors with different experience levels \parencite{girres_quality_2010}. 

\section{Temporal data quality and the evolution of OSM data}
\section{OSM production and use in humanitarian contexts}
\section{Conclusions}

