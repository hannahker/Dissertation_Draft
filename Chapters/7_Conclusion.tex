\chapter{Conclusion}
\label{chapterlabel7}

This research employs a comparative case study approach to investigate data production and maintenance in humanitarian mapping campaigns. This work responds to the need to more rigorously consider dimensions of temporal data quality in OSM, particularly within humanitarian mapping contexts. We focused specifically on humanitarian campaigns in Port au Prince, Bangui, Tacloban, and Kathmandu; and compare against mapping in Heidelberg as a reference. The recently developed OSHDB API was applied to efficiently process and filter large volumes of historical OSM data. In addition to the key results summarized below, this research builds off of \textcite{quattrone_work_2017} and offers a methodological approach for empirically assessing data maintenance in OSM.  

Following the framework set out by \textcite{dittus_mass_2017}, we classified Bangui and Heidelberg as mission-style campaigns; and Tacloban, Kathmandu, and Port au Prince as event-style campaigns. Our humanitarian case studies differed from our Heidelberg reference in both the overall and daily volume of new contributions to OSM, with the humanitarian campaigns showing significantly more data added over shorter time frames. 

Our results also show that the data produced during our selected humanitarian campaigns has been poorly maintained over time when compared against the maintenance levels seen in Heidelberg. Across all humanitarian case studies, the majority of data produced during the mapping campaign has not been modified or deleted after four years. This finding suggests that the OSM data in these areas is at a greater risk of becoming out of date. Thus, the humanitarian mapping community may need to consider developing formal mechanisms or incentives for ensuring that the data produced during mapping campaigns is maintained over time, allowing for it to be a lasting resource for the affected communities. 


