\chapter{Introduction}
\label{chapterlabel1}

Accurate and up-to-date geospatial data is an important resource that enables an effective response to a humanitarian crisis \parencite{cowan_geospatial_2011, poser_volunteered_2010, soden_infrastructure_2016, zook_volunteered_2010}. Such data can be critical for functions such as distributing aid, identifying affected regions, and coordinating response between humanitarian organizations \parencite{soden_infrastructure_2016}. However, as many regions with humanitarian need do not have official geospatial data, alternative sources of crowdsourced data are often needed \parencite{zook_volunteered_2010}. 

OpenStreetMap (OSM) is a valuable source of open and freely available geospatial data that is often used in humanitarian operations \parencite{palen_success_2015, soden_infrastructure_2016}. However, as OSM is an example of what \textcite{goodchild_citizens_2007} terms “volunteered geographic information” (VGI), it does not have any formal mechanisms for quality control and so may be considered less trustworthy by data users. 

This work furthers the existing body of literature on data quality in OSM. Situated within the humanitarian context, this work considers the temporality that is inherent to geospatial data production in OSM. Just as real-world geographic features and their associated attributes change over time, so must their digital representations within OSM. The concept of temporal data quality is explored by investigating practices of data maintenance, which is considered to be the necessary process by which data is kept up-to-date. 

Data maintenance is particularly relevant in humanitarian mapping contexts due to the unique modes of data production in this domain. The response effort following the 2015 Kathmandu earthquake demonstrates how event-based humanitarian mapping campaigns have been able to quickly produce large volumes of up-to-date geospatial data \parencite{soden_infrastructure_2016}. However, much of this data is produced by remote volunteers \parencite{eckle_quality_2015} who may not be invested in the quality of the data over time. This primarily remote nature of contribution coupled with the large volume of data may mean that the OSM data produced during humanitarian mapping campaigns is at a risk of quickly becoming out of date. However, this has yet to be empirically evaluated. While one cannot deny the value of this data in the wake of a crisis, the humanitarian mapping community also acknowledges the importance of longer-term sustainability of this data and its value as a community resource after a crisis subsides \parencite{soden_crowdsourced_2014}. 

In this research, I aim to explore the characteristics of geospatial data production during selected humanitarian mapping campaigns in OSM, and the extent to which this data is maintained following each campaign. It is hoped that the results of this analysis will contribute to a greater understanding of the quality of data that is produced in humanitarian mapping efforts, particularly relating to the data’s ongoing temporal accuracy. I employ a comparative case study approach whereby I investigate four selected humanitarian mapping campaigns and one reference period of mapping activity in a region of known high data quality. Humanitarian case studies are selected from Port au Prince, Bangui, Tacloban, and Kathmandu; and Heidelberg, Germany is selected as the reference case study. It is intended for the results of this analysis to provide a foundation for future work in this emerging research domain. 

This work explores the following specific research questions: 

\begin{itemize}
    \item \textbf{RQ 1:} What are the characteristics of data production in the selected humanitarian mapping campaigns and how does this compare with the reference case study?
    \item \textbf{RQ 2:} To what extent is the data produced during the selected campaigns maintained over time and how does this compare with the reference case study? 
    \item \textbf{RQ 3:} What insight do these results offer into potential relationships between characteristics of data production and levels of data maintenance in each of the case studies? 
\end{itemize}

These research questions are addressed throughout this document as follows: In \textbf{Chapter 2}, I review existing academic literature relating to OSM data quality and humanitarian applications. I critically examine this past literature to identify a key research gap that this work addresses. In \textbf{Chapter 3}, I provide a brief description of the OSM data model and outline relevant computational challenges in processing historical OSM data. I highlight the OSHDB framework \parencite{raifer_oshdb_2019} as the state-of-the-art in managing this data. I outline my methodology in \textbf{Chapter 4}, describing my case study approach, techniques in data collection and processing, and procedures for investigating data production and maintenance. I present the results of this analysis in \textbf{Chapter 5} and discuss their significance and limitations in \textbf{Chapter 6}. I conclude this work in \textbf{Chapter 7} and provide recommendations for future research efforts. 

