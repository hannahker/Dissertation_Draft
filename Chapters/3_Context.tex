\chapter{Case Studies and Data Description}
\label{chapterlabel3}

This chapter describes the characteristics of each case study under investigation and the OSM data that is used in subsequent analyses. We also conclude with a discussion of relevant ethical considerations. 

\section{Case studies}

The scope of this work is limited to three distinct case studies. We focus on humanitarian mapping activities in 1) Port au Prince, Haiti, following the 2010 earthquake, 2) Tacloban, Philippines, following the 2015 typhoon, and 3) Kutupalong, Bangladesh, which is the site of an ongoing refugee crisis. The following criteria were used to select case studies: 

\begin{itemize}
    \item Mapping in response to a humanitarian need, such as a natural disaster, disease outbreak, or political crisis.
    \item Mapping in a distinct and well-defined subnational geographic area.
	\item Sufficient volume of mapping activity on OSM, as defined by volume of unique contributors and volume of edits over time. 
    \item Sufficient documentation of the humanitarian mapping response, through content on pages such as the OSM Wiki, HOT Projects page, HOT Tasking Manager. 
\end{itemize}

The details for each of these case studies is described below. 



\section{Data description}



    


